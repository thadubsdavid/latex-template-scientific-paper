% bei einem vierseitigen Inhaltsverzeichnis darf die Nummerierung mit 1,2, .... erst hier beginnen, in den folgenden Kapiteln entfällt der Befehl
% bei einem vierseitigen Inhaltsverzeichnis diesen Befehl nicht auskommentieren
%\pagenumbering{arabic}

\section{Einleitung}\label{chap:einleitung}

Der grobe Aufbau und die Gliederung dieses Dokumentes entspricht dem typischen Aufbau einer studentischen Arbeit.

\subsection{Motivation}\label{sec:motivation}
Hier soll stehen, warum die in der Arbeit behandelten Konzepte, Betrachtungen und Lösungen gebraucht werden und praktisch relevant sind.

\subsection{Problemstellung und Zielsetzung}\label{sec:zielsetzung}
Was soll mit der Arbeit erreicht werden?

\subsection{Aufbau der Arbeit}\label{sec:aufbau}
Hier steht, was in den restlichen Kapiteln behandelt wird. Die Einleitung endet mit diesem Abschnitt.

\cleardoublepage

%%%%%%%%%%%%%%% PSA Struktur %%%%%%%%%%%%%%%%%%%%%
\begin{comment}
\section{Einleitung}\label{chap:einleitung}

\subsection{Motivation}\label{sec:motivation}

\subsection{Problemstellung und Zielsetzung}\label{sec:zielsetzung}

\subsection{Unternehmensvorstellung}\label{sec:unternehmen}

\subsection{Aufbau der Arbeit}\label{sec:aufbau}
\end{comment}