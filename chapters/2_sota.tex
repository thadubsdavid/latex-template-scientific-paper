\section{Stand der Forschung und Technik}\label{chap:sota}
Hier steht alles, was es schon gibt. Relevante wissenschaftliche Literatur ist an geeigneter Stelle einzubeziehen. Es sollte deutlich werden, wo die Lücke für das behandelte Thema ist, also wie es sich abgrenzt. Hier sollte auch die Einordnung in das jeweilige Forschungsgebiet erfolgen. Bei manchen Themen bietet es sich auch an, ein separates Grundlagenkapitel einzuführen. Zum Teil ist ein weiteres Kapitel zur Anforderungsanalyse sinnvoll.\\
Zur Recherche wissenschaftlicher Literatur gibt es viele Möglichkeiten. Neben Werken aus Bibliotheken und Webangeboten entsprechender Produkte, Lösungen oder Forschungsprojekte finden sich v.\,a. Konferenzbände aus dem Informatikbereich in den Online-Datenbanken der großen Publisher, z.\,B. ACM Digital Library \cite{acmdl}, IEEE Xplore \cite{ieeeexplore} oder SpringerLink \cite{springerlink}. Es besteht die Möglichkeit, einige dieser Datenbanken über das Hochschul-Netz kostenlos zu nutzen. Meta-Suchdienste, wie Google Scholar \cite{googlescholar} oder CiteSeerX \cite{citeseerx} stellen eine weitere Möglichkeit dar. Zum Teil sind auch aktuelle Informationen auf den privaten Webseiten entsprechender Forschergruppen zu finden.


\cleardoublepage

%%%%%%%%%%%%%%% PSA Struktur %%%%%%%%%%%%%%%%%%%%%
\begin{comment}
\section{Theoretische Grundlagen}\label{chap:sota}
\subsection{Einführung Projektmanagement}\label{sec:introductionpm}


\subsection{Einführung <Schwerpunkt>}\label{sec:introductiondevops}



\begin{table}[ht]
	\centering
	\caption{Einbindung von Abkürzungen SOTA mit  \texttt{glossaries}}
	\label{tab:glossariesusage}
	\begin{tabular}{ll} \toprule
		\textbackslash{}gls\{\} & Normale Einbindung (lange Form beim ersten Auftreten)\\
		\textbackslash{}Gls\{\} & Einbindung mit großem Anfangsbuchstaben\\
		\textbackslash{}glspl\{\} & Pluralform (mit s oder wie angegeben)\\
		\textbackslash{}glslink\{\}\{\} & Anzeige eines beliebigen anderen Textes\\
		\bottomrule
	\end{tabular}
\end{table}
\end{comment}
