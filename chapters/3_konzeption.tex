\section{Konzeption}\label{chap:konzeption}
In diesem Kapitel erfolgt die Darstellung des neuen, eigenen Konzeptes, welches
es in dieser Form bisher noch nicht gibt. Dabei muss nachvollziehbar sein, wie die
erarbeitete eigene Leistung sich in das Thema und das Forschungsgebiet einordnet.
Alles, was in der Arbeit beschrieben wird, muss einen Bezug zum Thema bzw. zum
vorgestellten Konzept haben. Werden für das Konzept Teile anderer Lösungen bzw.
Ansätze verwendet oder weiterentwickelt, so ist dies deutlich von der eigenen
Leistung abzugrenzen.


\cleardoublepage

%%%%%%%%%%%%%%% PSA Struktur %%%%%%%%%%%%%%%%%%%%%
\begin{comment}
\section{Projektinitiierung}\label{chap:projektinit}
\subsection{Unternehmenskontext}\label{sec:unternehmen}
Hier steht, was in den restlichen Kapiteln behandelt wird. Die Einleitung endet mit diesem Abschnitt.

\subsection{Projektauftrag}\label{sec:projektauftrag}
Projektsteckbrief/auftrag beschreiben und in Tabellenform dem Anhang hinzufügen 

\subsection{Stakeholderanalyse}\label{sec:stakeholderanalyse}
\subsubsection{Stakeholder-Identifikation und -Engagement}\label{sec:stakeholderident}
Darstellung der Workshop und Teamstruktur 

\subsubsection{Personas}\label{sec:personas}
Vorstellung des Kundensegments und einer von drei Personas 
\end{comment}