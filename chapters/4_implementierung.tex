\section{Implementierung}\label{chap:implementierung}
Hier soll eine kurze Beschreibung der Herausforderungen und der wichtigen
Eckdaten der Implementation des zuvor präsentierten Konzeptes vorgenommen
werden. Gegebenenfalls wird hier noch auf die Evaluation der Implementation
eingegangen und deren Messwerte interpretiert.

\begin{lstlisting}[language=java, numbers=none, caption=Exemplarisches Beobachter-Muster]
public enableProcessDisruptionSubject: Subject<THaDisruption> = new Subject();  
public enableProcessDisruptionObservable: Observable<THaDisruption> = 		  this.enableProcessDisruptionSubject.asObservable(); 
\end{lstlisting}

\cleardoublepage

%%%%%%%%%%%%%%% PSA Struktur %%%%%%%%%%%%%%%%%%%%%
\begin{comment}
\section{Projektplanung}\label{chap:projektplanung}

\subsection{Projektstrukturplan}\label{sec:strukturplan}
Hier steht, was in den restlichen Kapiteln behandelt wird. Die Einleitung endet mit diesem Abschnitt.

\subsection{Projektzeitplan}\label{sec:zeitplan}
Hier steht, was in den restlichen Kapiteln behandelt wird. Die Einleitung endet mit diesem Abschnitt.

\subsection{Risikomanagement}\label{sec:risikomgmt}
Hier steht, was in den restlichen Kapiteln behandelt wird. Die Einleitung endet mit diesem Abschnitt.
\end{comment}



