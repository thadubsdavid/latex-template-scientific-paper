\documentclass[DIV=10, parskip=full*, headings=big, twopage, pdftex, 11pt, cleardoublepage=plain, 
				bibliography=totoc, listof=totoc, BCOR=12mm,
				numbers=noenddot]{article}
%\documentclass[
%scrbook = For thesis a book structure. By default is a double-sided document leading to several empty resp. white pages after some chapters because new chapters are starting on odd pages.
%article = For TDR structure. This option can remove "trailing" white pages 
%]{class}

\usepackage[ngerman]{babel}

\usepackage{helvet}
%\usepackage[a4paper, left=4cm, right=2cm, top=2cm, bottom=2cm]{geometry}
\usepackage[utf8]{inputenc}
\usepackage[T1]{fontenc} % for correct guillemets, http://en.wikibooks.org/wiki/LaTeX/Internationalization#German
\usepackage{lmodern} % to avoid rastered fonts, http://www.komascript.de/node/699
\usepackage{scrhack}

\usepackage[babel,german=quotes]{csquotes}
\usepackage[style=authoryear,backref=true,minnames=3,abbreviate=false]{biblatex}

%Paket laden
\usepackage[automake, acronym, toc]{glossaries}
%\usepackage[
%nonumberlist, %keine Seitenzahlen anzeigen
%acronym,      %ein Abkürzungsverzeichnis erstellen
%toc,          %Einträge im Inhaltsverzeichnis
%section]      %im Inhaltsverzeichnis auf section-Ebene erscheinen
%{glossaries}
%\makeglossaries
\makeglossaries
\newglossaryentry{RCP}{name={RCP},description={Rich Client Platform}}
\newglossaryentry{XPath}{name={XPath},description={XML Path Language}}
\newglossaryentry{XHTML}{name={HTML},description={Extensible Hypertext Markup Language}}
\newglossaryentry{SVN}{name={SVN},description={Subversion}}

\usepackage{booktabs}

\usepackage{typearea}

%Abk�rzungsverzeichnis
\usepackage{nomencl}
\renewcommand{\nomname}{Abkürzungsverzeichnis}
\setlength{\nomlabelwidth}{.20\hsize}
\renewcommand{\nomlabel}[1]{#1 \dotfill}

\makenomenclature

%Schriften
\usepackage[sc]{mathpazo}
\linespread{1.05}
\usepackage{microtype}
\usepackage{amssymb}

\usepackage{changebar}

% Palatino als Brotschrift
%\fontfamily{phv}\selectfont


% AvantGarde f�r �berschriften
\renewcommand{\familydefault}{\sfdefault} % default für alle seriflosen bei article document structure
\renewcommand{\sfdefault}{phv} % default f�r alles serifenlose --> AvantGarde
%\setkomafont{subtitle}{\fontfamily{phv}\fontseries{m}\selectfont}
%\setkomafont{subject}{\fontfamily{phv}\fontseries{m}\selectfont}
%\setkomafont{section}{\sffamily\footnotesize}
%\setkomafont{pagenumber}{\sffamily\footnotesize}

\usepackage[onehalfspacing]{setspace}
%%%%%%%%%%%%% FONT TYPES %%%%%%%%%%%%%%
%ptm	Times
%phv	Helvetica
%pcr	Courier
%pbk	Bookman
%pag	Avant Garde
%ppl	Palatino
%bch	Charter
%pnc	New Century Schoolbook
%put	Utopia

% bold letters in equation
\usepackage{amsfonts}

%Reference https://www.physicsforums.com/threads/mathbb-stuff-isnt-working.544078/


\usepackage{xcolor}
\definecolor{darkgray}{gray}{0.2} 
\color{darkgray}
\definecolor{lightgrey}{rgb}{0.9, 0.9, 0.9} 
\definecolor{txtgreen}{rgb}{0.549, 0.749, 0.149}

\usepackage{amssymb, amsmath}
\usepackage[final]{pdfpages}

\usepackage{subfigure}
\usepackage{graphicx}
\usepackage{wrapfig}
\usepackage{comment}

\usepackage{flafter}
\usepackage{multicol}
\usepackage{multirow}
\usepackage{tabulary}
\usepackage{listings} % zum Einbringen von Code mit Syntaxhightlighting
\usepackage{blindtext}
\usepackage[perpage, ragged]{footmisc}

\definecolor{darkred}{rgb}{0.85, 0.3, 0.4}
\usepackage{courier}
\lstset{
		basicstyle=\small\ttfamily,
		keywordstyle=\color{blue},
		stringstyle=\color{darkred},
		commentstyle=\tiny\color{gray}
}
% voreingestellte Programmiersprache: Java
\lstset{
        language=Java, 
        breaklines=true, 
        breakautoindent=true, 
        breakatwhitespace=true, 
        numbers=left, 
        numberstyle=\tiny,
        extendedchars=true,
        rulesepcolor=\color{red},
        frame=lines,
        keywordstyle=\color{blue},
        commentstyle=\color{gray},
        stringstyle=\color{darkred},
        emph={double,bool,int,unsigned,char,true,false,void},
				emphstyle=\color{blue},
				emph={Assert,Test},
				emphstyle=\color{red},
				emph={[2]\using,\#define,\#ifdef,\#endif}, emphstyle={[2]\color{blue}},
		basicstyle=\small\ttfamily,
		showstringspaces=false,
		commentstyle=\tiny}
		

\usepackage[
	pdftex,
	hyperfootnotes=false,
	bookmarks,
	pdfpagelabels=true,
	plainpages=false,
	breaklinks]{hyperref} % f�r Verlinkungen in der PDF (extern und intern)


%colorlinks=false needs to be true in order to apply color
%\definecolor{lc}{cmyk}{0.6,0.16,1,0.015} % for screen output
%\definecolor{lc}{cmyk}{0.05,1.0,0.8, 0.0} % -red for screen output
%\definecolor{lc}{cmyk}{0.05,1.0,0.8, 0.2} % -red-dark1 for screen output
%\definecolor{lc}{cmyk}{0.05,1.0,0.8, 0.4} % -red-dark2 for screen output
%\definecolor{lc}{cmyk}{0.05,1.0,0.8, 0.6} % -red-dark3 for screen output
%\definecolor{lc}{cmyk}{1.0,0.0,0.55,0.2} % -green for screen output
%\definecolor{lc}{cmyk}{1.0,0.0,0.0,0.0} % -blue for screen output
%\definecolor{lc}{cmyk}{1.0,0.3,0.8,0.3} % -yellow for screen output
% turn on black link-color and set colorlink on false
%\definecolor{lc}{cmyk}{75,68,67,90} % -black for screen output
\newcommand{\printoutput}{\definecolor{lc}{cmyk}{0,0,0,1}} % for print output

%turn off link coloring by comment outlinkcolor downwards
\hypersetup{
	pdftoolbar=true,
	bookmarksopen,
	bookmarksnumbered=false,
	bookmarksopenlevel=1,
	pdfdisplaydoctitle,	
	colorlinks=false,
	linkcolor=lc,
	citecolor=lc,
	filecolor=lc,
	menucolor=lc,
	urlcolor=lc	
}
%\graphicspath{{images/}}

\usepackage[automark, footsepline]{scrlayer-scrpage}

%%%%%%%%%% Tabellenanpassungen %%%%%%%%%%%%%%%%%%
\setlength{\tabcolsep}{3pt}
\renewcommand{\arraystretch}{2}

\usepackage{tabularx}
\usepackage{array}
\usepackage{colortbl}
\usepackage{longtable}
\usepackage{caption}
\captionsetup{font=small}
\captionsetup{labelfont=bf}
\renewcommand{\tabularxcolumn}[1]{m{#1}} % Zellen von Tabellen immer vertikal zentrieren

%%%%%%%%%%%%%% Listen %%%%%%%%%%%%%%%%%%%%%
\usepackage{mdwlist}
\usepackage{setspace}
\renewcommand{\labelitemi}{\textperiodcentered}%{\sqbullet} %{\textperiodcentered}
\usepackage{enumitem}
\setlist{noitemsep}
\renewcommand{\labelenumi}{(\arabic{enumi})}

%%%%% Aufz�hlungsstyle %%%%%%%%
\newcommand{\myitem}[1]{\textsf{\textbf{#1}}.\hspace{1em}}

%%%%%%%%%% Pagina und Kolumnentitel %%%%%%%%%%%%%
\pagestyle{scrheadings}
\ihead{\IUMStype}
\ohead{\IUMSauthor}

\ofoot{\pagemark}
\ifoot{\SCMTsecretdocument}

\KOMAoptions{headsepline=0.5pt}
\KOMAoptions{footsepline=0pt}

\setcounter{secnumdepth}{3} % Tiefe der �berschriften, die Nummern bekommen
\setcounter{tocdepth}{2}    % Inhaltsverzeichnis bis Tiefe 3

%%%%%%%%%%%%%%%%%%%% Warnings %%%%%%%%%%%%%%%%%%%%%%%%
\pdfsuppresswarningpagegroup=1

%%%%%%%%%%%%%%%%%%%% TODO Notes %%%%%%%%%%%%%%%%%%%%%%%%
\usepackage[colorinlistoftodos]{todonotes} % TODO notes

\makeglossaries


% MetaData
\newcommand{\IUMSauthor}{Vorname, Name}
\newcommand{\IUMSmatNr}{XXX-XXXXXXX}

\newcommand{\IUMStype}{Art der Arbeit}%-BITTE ART UND BEZEICHNUNG ENTSPRECHEND DEN ANGABEN LEISTUNGSNACHWEISE IM EIS VERWENDEN} % Master-Thesis/ Bachelor-Thesis
\newcommand{\IUMStitle}{Langer Titel für die wissenschaftliche Arbeit}
\newcommand{\IUMSsubtitle}{IM RAHMEN DES PROJEKT-KOMPETENZ-STUDIUMS\\ \vspace{0.25em} ZUM }
\newcommand{\IUMSsubmissionMonth}{21. Dezember 2017} % z.B. \today
\newcommand{\IUMSsubmissionDate}{21. Dezember 2017} % z.B. \today
\newcommand{\SCMTstartevent}{30. Oktober 2020} % z.B. \today
\newcommand{\SCMTendevent}{30. Oktober 2020} % z.B. \today
\newcommand{\SCMTdash}{ - }
\newcommand{\SCMTlocation}{Stuttgart}
\newcommand{\SCMTstudyprogramlarge}{WIRTSCHAFTSINFORMATIK M.SC. }
\newcommand{\SCMTstudyprogramregular}{Wirtschaftsinformatik M.Sc. }
\newcommand{\SCMTsupervisor}{Titel, Vorname, Name}
\newcommand{\companyname}{<Firma> }
\newcommand{\companysupervisor}{Titel, Vorname, Name}
\newcommand{\companyprojectleader}{<Titel>/<Name>/<Position>/<Abteilung>}
\newcommand{\SCMTdatetime}{Ort, Datum}
\newcommand{\SCMTsignature}{Unterschrift}
\newcommand{\SCMTsecretdocument}{Vertrauliches Dokument}
% End MetaData

% Uncomment for print version
%\printoutput

% Bib-File
%\bibliographystyle{stylename}
\bibliography{bibliography/literatur}

\begin{document}

\hypersetup{pageanchor=false}
\begin{titlepage}
\begin{figure}[htbp]
  \begin{minipage}[t][2cm][c]{\textwidth/2+1.7cm}
	
		\includegraphics[height=4em]{_includes/images/no-logo.png}  

  \end{minipage}
  \begin{minipage}[t][3cm][c]{\textwidth/2-1.7cm}
	\begin{flushright}
  %	\includegraphics[height=5em]{_includes/images/<>.jpeg}
	\end{flushright}
  \end{minipage}
\end{figure}



\vspace{3em}
\begin{center}
	\large{\textsf{\IUMStype}}
\end{center}

\vspace{2em}
\begin{figure}[htbp]
\centering
\includegraphics[height=9em]{_includes/images/no-logo.png}
\end{figure}

\begin{center}
\vspace{2em}
\Huge{\textbf{\textsf{\IUMStitle}}}


\large{\emph{\textsf{\IUMSsubtitle\SCMTstudyprogramlarge}}}
\vspace{\fill}
\end{center}

\vspace{1.5em}
\normalsize 
\begin{center}
	\textbf{\textsf{Autor:}}\\ \vspace{0.25em}
	\IUMSauthor\\ \vspace{0.25em}
	Mat.-Nr.: \IUMSmatNr %Durchführung ist Semester Jahrgang "Winf 05"	
\end{center}
%\begin{flushleft}
%
%\begin{figure}[b]
%\centering
%\begin{minipage}[b]{\textwidth/2-0.1cm+0.85cm}
%\begin{flushleft}
%\IUMSsubmissionMonth\\ \vspace{1em}
%\IUMSauthor\\ \vspace{0.25em}
%Mat.-Nr.: \IUMSmatNr
%\end{flushleft}
%\end{minipage}
%\begin{minipage}[b]{\textwidth/2-0.1cm-0.85cm}
%\begin{flushleft}
%\textbf{\textsf{Betreuer}}\\ \vspace{0.25em}
%\IUMSsupervisor\\
%\vspace{1em}
%\textbf{\textsf{Verantwortl. Hochschullehrer}}\\ \vspace{0.25em}
%Akad. Grad Vorname Nachname
%\end{flushleft}
%\end{minipage}
%\end{figure}

%\end{flushleft}

\end{titlepage}

%\thispagestyle{empty}
%\cleardoublepage

\hypersetup{pageanchor=true}
\pagenumbering{roman}

\hypersetup{pageanchor=false}

\begin{titlepage}
\begin{figure}[htbp]
  \begin{minipage}[t][2cm][c]{\textwidth/2+1.7cm}
	
		%\includegraphics[height=4em]{_includes/images/scmt-logo.png}  

  \end{minipage}
  \begin{minipage}[t][3cm][c]{\textwidth/2-1.7cm}
	\begin{flushright}  	
  %	\includegraphics[height=5em]{_includes/images/<>.jpeg}
	\end{flushright}
  \end{minipage}
\end{figure}



%\vspace{5em}
%\large{\textsf{\centerline{\IUMStype}}}

%\vspace{2em}
%\begin{figure}[htbp]
%\centering
%\includegraphics[height=5em]{_includes/images/witzenmann-logo.jpg}
%\end{figure}

\begin{center}
\vspace{2em}
\Huge{\textbf{\textsf{\IUMStitle}}}
\end{center}

\vspace{2em}

\normalsize 
\begin{center}
\textbf{\textsf{Autor:}}\\ \vspace{0.25em}
\IUMSauthor\\ %\vspace{0.25em}

\vspace{5em}

\textbf{\textsf{Zeitraum der Bearbeitung:}}\\ \vspace{0.25em}
\SCMTstartevent \SCMTdash \SCMTendevent \\ \vspace{1em}
\end{center}

\begin{flushleft}

\begin{figure}[b]
\centering
\begin{minipage}[b]{\textwidth/2-0.1cm+0.85cm}
\begin{flushleft}
\begin{center}
\textbf{\textsf{Unternehmensbetreuer:}}\\ \vspace{0.25em}
\companysupervisor\\
\end{center}
\end{flushleft}
\end{minipage}
\begin{minipage}[b]{\textwidth/2-0.1cm-0.85cm}
\begin{flushleft}
\begin{center}
\textbf{\textsf{Hochschulbetreuer:}}\\ \vspace{0.25em}
\SCMTsupervisor\\
\end{center}
\end{flushleft}
\end{minipage}
\end{figure}

\end{flushleft}

\end{titlepage}

%\thispagestyle{empty}
%\cleardoublepage

\hypersetup{pageanchor=true}

\thispagestyle{empty}
%\vspace*{\fill}

\section*{EIGENSTÄNDIGKEITSERKLÄRUNG}
Hiermit versichere ich, die vorliegende Arbeit im Rahmen des Projekt-Kompetenz-Studiums zum \SCMTstudyprogramregular an der Steinbeis-Hochschule Berlin selbständig verfasst und keine anderen als die angegebenen Hilfsmittel benutzt zu haben.\\[0.5em]
Ferner versichere ich, dass Stellen, die anderen Werken – auch elektronischen Medien – dem Wortlaut oder Sinn nach entnommen wurden sowie übernommene Zeichnungen, bildliche Darstellung, Skizzen und dergleichen, unter Angabe der Quelle als Entlehnung kenntlich gemacht worden sind.

%\SCMTlocation, den \SCMTendevent\\
\vspace*{4em}
\begin{flushleft}
	\makebox[.4\textwidth]{\hrulefill}\hfill \makebox[.4\textwidth]{\hrulefill}\\
	\makebox[.4\textwidth]{\SCMTdatetime}\hfill
	\makebox[.4\textwidth]{Vorname Nachname}\\
\end{flushleft}




\thispagestyle{empty}
%\vspace*{\fill}

\section*{SPERRVERMERK (NUR FÜR PSA UND THESIS)}
Diese wissenschaftliche Arbeit unterliegt der Geheimhaltung. Sie darf nicht ohne Genehmigung des Unternehmens Dritten gezeigt bzw. veröffentlicht werden.
\vspace*{4em}
\begin{flushleft}
	\makebox[.4\textwidth]{\hrulefill}\hfill \makebox[.4\textwidth]{\hrulefill}\\
	\makebox[.4\textwidth]{\SCMTdatetime}\hfill
	\makebox[.4\textwidth]{Studierender}\\
\end{flushleft}
\vspace*{4em}
\begin{flushleft}
	\makebox[.4\textwidth]{\hrulefill}\hfill \makebox[.4\textwidth]{\hrulefill}\\
	\makebox[.4\textwidth]{\SCMTdatetime}\hfill
	\makebox[.4\textwidth]{Unternehmensbetreuer}\\
\end{flushleft}
\vspace*{4em}
\begin{flushleft}
	\makebox[.4\textwidth]{\hrulefill}\hfill \makebox[.4\textwidth]{\hrulefill}\\
	\makebox[.4\textwidth]{\SCMTdatetime}\hfill
	\makebox[.4\textwidth]{Hochschulbetreuer}\\
\end{flushleft}




\thispagestyle{empty}
%\vspace*{\fill}

\section*{GENEHMIGUNG (NUR FÜR PROJEKTSPEZIFIKATION)}
Die Projektspezifikation wurde Herr/Frau \companyprojectleader, \companyname als Projektleiter in der vorliegenden Form zur Abzeichnung vorgelegt.\\[0.5em]
Kenntnis genommen und Einverständnis erklärt:

\vspace*{4em}
\begin{flushleft}
	\makebox[.4\textwidth]{\hrulefill}\hfill \makebox[.4\textwidth]{\hrulefill}\\
	\makebox[.4\textwidth]{\SCMTdatetime}\hfill
	\makebox[.4\textwidth]{Studierender}\\
\end{flushleft}
\vspace*{4em}
\begin{flushleft}
	\makebox[.4\textwidth]{\hrulefill}\hfill \makebox[.4\textwidth]{\hrulefill}\\
	\makebox[.4\textwidth]{\SCMTdatetime}\hfill
	\makebox[.4\textwidth]{Unternehmensbetreuer}\\
\end{flushleft}
\vspace*{4em}
\begin{flushleft}
	\makebox[.4\textwidth]{\hrulefill}\hfill \makebox[.4\textwidth]{\hrulefill}\\
	\makebox[.4\textwidth]{\SCMTdatetime}\hfill
	\makebox[.4\textwidth]{Hochschulbetreuer}\\
\end{flushleft}




\linespread{1.13}

% \pdfbookmark[1]{Inhaltsverzeichnis}{toc}
\tableofcontents					% create a toc
%\addcontentsline{toc}{chapter}{Inhaltsverzeichnis}
% wenn eine Seite beim Inhaltsverzeichnis fehlt (also zB drei statt vier Seiten lang ist):
% \thispagestyle{empty}
\cleardoublepage

\linespread{1}
%\pagenumbering{arabic}
%
%\include{acronyms/acronymsTest}
% bei vierseitigen Inhaltsverzeichnis erst im Kapitel 1
%\pagenumbering{arabic}

%%%%% ABBILDUNGSVERZEICHNIS %%%%%
%\addcontentsline{toc}{chapter}{Abbildungsverzeichnis}
\addcontentsline{toc}{section}{\listfigurename}\listoffigures
\cleardoublepage

%%%%% TABELLENVERZEICHNIS %%%%%
\addcontentsline{toc}{section}{\listtablename}\listoftables
\cleardoublepage

%%%%%  GLOSSARY  %%%%%%
%\printglossaries[type=acronym]
\printglossary[title={Abkürzungsverzeichnis}] %Generate List of Abbreviations
\cleardoublepage

%%%%% QUELLCODEVERZEICHNIS %%%%%
%\addcontentsline{toc}{section}{\lstlistlistingname}
%\renewcommand\lstlistlistingname{Programmcodeverzeichnis}
%\lstlistoflistings

\newpage
\cleardoublepage

\linespread{1}
\pagenumbering{arabic}
%%%%% INHALTSKAPITEL %%%%%
% bei einem vierseitigen Inhaltsverzeichnis darf die Nummerierung mit 1,2, .... erst hier beginnen, in den folgenden Kapiteln entfällt der Befehl
% bei einem vierseitigen Inhaltsverzeichnis diesen Befehl nicht auskommentieren
%\pagenumbering{arabic}

\section{Einleitung}\label{chap:einleitung}

Der grobe Aufbau und die Gliederung dieses Dokumentes entspricht dem typischen Aufbau einer studentischen Arbeit.

\subsection{Motivation}\label{sec:motivation}
Hier soll stehen, warum die in der Arbeit behandelten Konzepte, Betrachtungen und Lösungen gebraucht werden und praktisch relevant sind.

\subsection{Problemstellung und Zielsetzung}\label{sec:zielsetzung}
Was soll mit der Arbeit erreicht werden?

\subsection{Aufbau der Arbeit}\label{sec:aufbau}
Hier steht, was in den restlichen Kapiteln behandelt wird. Die Einleitung endet mit diesem Abschnitt.

\cleardoublepage

%%%%%%%%%%%%%%% PSA Struktur %%%%%%%%%%%%%%%%%%%%%
\begin{comment}
\section{Einleitung}\label{chap:einleitung}

\subsection{Motivation}\label{sec:motivation}

\subsection{Problemstellung und Zielsetzung}\label{sec:zielsetzung}

\subsection{Unternehmensvorstellung}\label{sec:unternehmen}

\subsection{Aufbau der Arbeit}\label{sec:aufbau}
\end{comment}
%\pagenumbering{arabic}
\section{Stand der Forschung und Technik}\label{chap:sota}
Hier steht alles, was es schon gibt. Relevante wissenschaftliche Literatur ist an geeigneter Stelle einzubeziehen. Es sollte deutlich werden, wo die Lücke für das behandelte Thema ist, also wie es sich abgrenzt. Hier sollte auch die Einordnung in das jeweilige Forschungsgebiet erfolgen. Bei manchen Themen bietet es sich auch an, ein separates Grundlagenkapitel einzuführen. Zum Teil ist ein weiteres Kapitel zur Anforderungsanalyse sinnvoll.\\
Zur Recherche wissenschaftlicher Literatur gibt es viele Möglichkeiten. Neben Werken aus Bibliotheken und Webangeboten entsprechender Produkte, Lösungen oder Forschungsprojekte finden sich v.\,a. Konferenzbände aus dem Informatikbereich in den Online-Datenbanken der großen Publisher, z.\,B. ACM Digital Library \cite{acmdl}, IEEE Xplore \cite{ieeeexplore} oder SpringerLink \cite{springerlink}. Es besteht die Möglichkeit, einige dieser Datenbanken über das Hochschul-Netz kostenlos zu nutzen. Meta-Suchdienste, wie Google Scholar \cite{googlescholar} oder CiteSeerX \cite{citeseerx} stellen eine weitere Möglichkeit dar. Zum Teil sind auch aktuelle Informationen auf den privaten Webseiten entsprechender Forschergruppen zu finden.


\cleardoublepage

%%%%%%%%%%%%%%% PSA Struktur %%%%%%%%%%%%%%%%%%%%%
\begin{comment}
\section{Theoretische Grundlagen}\label{chap:sota}
\subsection{Einführung Projektmanagement}\label{sec:introductionpm}


\subsection{Einführung <Schwerpunkt>}\label{sec:introductiondevops}



\begin{table}[ht]
	\centering
	\caption{Einbindung von Abkürzungen SOTA mit  \texttt{glossaries}}
	\label{tab:glossariesusage}
	\begin{tabular}{ll} \toprule
		\textbackslash{}gls\{\} & Normale Einbindung (lange Form beim ersten Auftreten)\\
		\textbackslash{}Gls\{\} & Einbindung mit großem Anfangsbuchstaben\\
		\textbackslash{}glspl\{\} & Pluralform (mit s oder wie angegeben)\\
		\textbackslash{}glslink\{\}\{\} & Anzeige eines beliebigen anderen Textes\\
		\bottomrule
	\end{tabular}
\end{table}
\end{comment}

\include{chapters/3_anforderungen}
\section{Konzeption}\label{chap:konzeption}
In diesem Kapitel erfolgt die Darstellung des neuen, eigenen Konzeptes, welches
es in dieser Form bisher noch nicht gibt. Dabei muss nachvollziehbar sein, wie die
erarbeitete eigene Leistung sich in das Thema und das Forschungsgebiet einordnet.
Alles, was in der Arbeit beschrieben wird, muss einen Bezug zum Thema bzw. zum
vorgestellten Konzept haben. Werden für das Konzept Teile anderer Lösungen bzw.
Ansätze verwendet oder weiterentwickelt, so ist dies deutlich von der eigenen
Leistung abzugrenzen.


\cleardoublepage

%%%%%%%%%%%%%%% PSA Struktur %%%%%%%%%%%%%%%%%%%%%
\begin{comment}
\section{Projektinitiierung}\label{chap:projektinit}
\subsection{Unternehmenskontext}\label{sec:unternehmen}
Hier steht, was in den restlichen Kapiteln behandelt wird. Die Einleitung endet mit diesem Abschnitt.

\subsection{Projektauftrag}\label{sec:projektauftrag}
Projektsteckbrief/auftrag beschreiben und in Tabellenform dem Anhang hinzufügen 

\subsection{Stakeholderanalyse}\label{sec:stakeholderanalyse}
\subsubsection{Stakeholder-Identifikation und -Engagement}\label{sec:stakeholderident}
Darstellung der Workshop und Teamstruktur 

\subsubsection{Personas}\label{sec:personas}
Vorstellung des Kundensegments und einer von drei Personas 
\end{comment}
\section{Implementierung}\label{chap:implementierung}
Hier soll eine kurze Beschreibung der Herausforderungen und der wichtigen
Eckdaten der Implementation des zuvor präsentierten Konzeptes vorgenommen
werden. Gegebenenfalls wird hier noch auf die Evaluation der Implementation
eingegangen und deren Messwerte interpretiert.

\begin{lstlisting}[language=java, numbers=none, caption=Exemplarisches Beobachter-Muster]
public enableProcessDisruptionSubject: Subject<THaDisruption> = new Subject();  
public enableProcessDisruptionObservable: Observable<THaDisruption> = 		  this.enableProcessDisruptionSubject.asObservable(); 
\end{lstlisting}

\cleardoublepage

%%%%%%%%%%%%%%% PSA Struktur %%%%%%%%%%%%%%%%%%%%%
\begin{comment}
\section{Projektplanung}\label{chap:projektplanung}

\subsection{Projektstrukturplan}\label{sec:strukturplan}
Hier steht, was in den restlichen Kapiteln behandelt wird. Die Einleitung endet mit diesem Abschnitt.

\subsection{Projektzeitplan}\label{sec:zeitplan}
Hier steht, was in den restlichen Kapiteln behandelt wird. Die Einleitung endet mit diesem Abschnitt.

\subsection{Risikomanagement}\label{sec:risikomgmt}
Hier steht, was in den restlichen Kapiteln behandelt wird. Die Einleitung endet mit diesem Abschnitt.
\end{comment}




\include{chapters/6_evaluation}
\section{Zusammenfassung und Ausblick}\label{chap:zusammenfassung}
Nochmal alles, was wichtig war wird hier erwähnt und der Bezug zur Zielstellung
und Motivation wird hergestellt. Es sollte auf wichtige Fragestellungen, die nicht
betrachtet wurden, aufmerksam gemacht und mögliche Ansätze bzw. Strategien
für weiterführende Arbeiten aufgezeigt werden.


\cleardoublepage

%%%%%%%%%%%%%%% PSA Struktur %%%%%%%%%%%%%%%%%%%%%
\begin{comment}
\section{Ausblick und Fazit}\label{chap:zusaus}

\subsection{Ausblick}\label{sec:ausblick}
Hier steht, was in den restlichen Kapiteln behandelt wird. Die Einleitung endet mit diesem Abschnitt.
Nochmal alles, was wichtig war wird hier erwähnt und der Bezug zur Zielstellung und Motivation wird hergestellt. Es sollte auf wichtige Fragestellungen, die nicht betrachtet wurden, aufmerksam gemacht und mögliche Ansätze bzw. Strategien für weiterführende Arbeiten aufgezeigt werden.

\subsection{Fazit}\label{sec:fazit}
Hier steht, was in den restlichen Kapiteln behandelt wird. Die Einleitung endet mit diesem Abschnitt.
Nochmal alles, was wichtig war wird hier erwähnt und der Bezug zur Zielstellung und Motivation wird hergestellt. Es sollte auf wichtige Fragestellungen, die nicht betrachtet wurden, aufmerksam gemacht und mögliche Ansätze bzw. Strategien für weiterführende Arbeiten aufgezeigt.
%zusätzliche leere Seite anfügen, da die Zusammenfassung nur eine Seite lang ist, aber der Anhang auf einer ungeraden Seite beginnen soll
%\thispagestyle{empty}
\end{comment}


\thispagestyle{empty}
%\nocite{*}

%%%%% BIBLIOGRAFIE %%%%%%
%\shorthandoff{"}
\printbibliography[title={Literaturverzeichnis}]
%\shorthandon{"}

%%%%%%%%%%%%%%%%%%%%%%%%%%% ANHANG %%%%%%%%%%%%%%%%%%%%%%%%%%%%%%%%%%%
%\begin{appendix}
\appendix
%\pagenumbering{roman}
\cleardoublepage
\section{Anhang}

%Diese Latexvorlage ist auf Basis der Vorlage des \gls{IUMS} Instituts der technischen Hochschule Karlsruhe entstanden und wird nun als eigene Version der Steinbeis Universität weiterentwickelt.

Der Anhang enthält Inhalte, die den Lesefluss im Text beeinflussen würden, beispielsweise sperrige Tabellen oder größere Quellcodebeispiele. In diesem Falle wird er für einige Hinweise genutzt. Eine Arbeit kann auch mehrere Anhänge enthalten, sie sind dabei gleichrangig mit Kapiteln, welche selbst wieder Abschnitte enthalten können.

\subsection{Schriftliche Ausarbeitung}
Beim Anfertigen einer wissenschaftlichen Arbeit mit \LaTeX{} wird viel Arbeit zur regelgerechten und konsistenten Einhaltung der Form bereits durch das Satzsystem geleistet. Darüber hinaus sind einige Hinweise zu beachten:
\begin{itemize}
	\item Die Angabe von Referenzen auf das Literaturverzeichnis erfolgt mit \texttt{\textbackslash{}cite\{\}}, wobei die Quellen einzeln \cite{knuth.literate84} oder gruppiert \cite{knuth.literate84,nm.epinjava,lamportlatex} auftreten können. Die Literaturangaben müssen im Bib\TeX{}-Format vorliegen. Zur Verwaltung kann z.\,B. \emph{JabRef} oder \emph{Mendeley} verwendet werden. Als Backend wird statt dem veralteten Bib\TeX{} das moderne \emph{Biber}\footnote{\url{http://biblatex-biber.sourceforge.net/}} eingesetzt.
	\item Anführungszeichen können komfortabel über das Package \texttt{csquotes} gesetzt werden, indem der in Anführungszeichen zu setzende Text mit \texttt{\textbackslash{}enquote\{\}} ausgezeichnet wird. Die Art der Anführungszeichen kann man in den Package-Optionen von csquotes ändern. Voreingestellt sind \enquote{diese} Anführungszeichen. Hervorhebungen \emph{dieser} Art bekommt man mit \texttt{\textbackslash{}emph\{\}}.
	\item Verzeichnisse wie das Abbildungsverzeichnis oder das Tabellenverzeichnis sollten erst ab einer Anzahl von mindestens drei Abbildungen bzw. Tabellen geführt werden.
	\item Die Arbeit sollte sich in Kapitel, Abschnitte und Unterabschnitte gliedern, wobei Gliederungspunkte ohne direkte Nachbarn zu vermeiden sind (also beispielsweise ein Abschnitt mit nur einem Unterabschnitt). Auch sollte nach jeder Überschrift Text folgen und nicht sofort die nächste Überschrift der nächstunteren Gliederungsebene. Mehr als drei Gliederungsebenen sollten grundsätzlich vermieden und nur in Absprache mit dem Betreuer verwendet werden.
	\item Verwendete Abbildungen, Tabellen und Codebeispiele sollten im Text referenziert und erklärt werden.
	\item Nach Möglichkeit sollten Abbildungen als Vektorgrafik eingebunden werden. Eine elegante Alternative ist auch die Erstellung von Abbildungen direkt in \LaTeX, z.\,B. mit dem Package \texttt{TikZ} (vgl. Abbildung \ref{fig:rpsls} auf Seite \pageref{fig:rpsls}). Lässt sich die Verwendung von Pixelbildern nicht vermeiden, wie etwa bei Screenshots oder Fotos mit natürlich beschaffenem Inhalt, ist auf eine hohe räumliche und Farbauflösung sowie auf eine hochwertige Bildkompression zu achten.
	\item Zahlen bis zwölf sollten als Wort geschrieben werden, danach als Ziffern, z.\,B. die Zahlen 23 und 42, falls dem keine relevanten andere Regeln im Wege stehen (etwa bei abgekürzten Maßeinheiten). Wichtiger ist es jedoch, dass die Konsistenz der Notation gewahrt bleibt, wenn mehrere Zahlen im gleichen Kontext erwähnt werden.
	\item Die Abstände innerhalb mehrgliedriger Abkürzungen, wie z.\,B., d.\,h. oder i.\,d.\,R. werden mit einem schmalen Leerzeichen gesetzt, das bei \LaTeX{} mit \textbackslash{}, erzeugt wird.
	\item Bei der Verwendung von Akronymen wie \Gls{RCP}, \gls{XHTML} oder \Gls{XPath} bzw. Glossareinträgen sind u.\,a. die Einbindungen aus Tabelle \ref{tab:glossariesusage} auf Seite \pageref{tab:glossariesusage} möglich.
	\item Die Absatzformatierung wird durch das Satzsystem sichergestellt und kann auf zwei verschiedene Arten erfolgen:\begin{enumerate}
		\item Einrückung (ab dem zweiten Absatz unter einer Überschrift) und kein extra Abstand zwischen den Absätzen (Vorgabeeinstellung)
		\item Abstand zwischen den Absätzen und keine Einrückung (Option \texttt{\textbackslash{}parskip} in der Dokumentklasse)
	\end{enumerate}
	\item Fußnoten werden mit \texttt{\textbackslash{}footnote}\footnote{Sie sollten sparsam eingesetzt werden.} erzeugt.
	\item Zitate fügt man mit \texttt{\textbackslash{}quote\{\}} ein: \begin{quote}The old computing was about what computers could do; the new computing is about what users can do.\end{quote}
	%\item TODO Inhalt des Inhaltsverzeichnis (Verzeichnisse), Reihenfolge der Bestandteile der Arbeit, Referenzen auf Labels, Figures, Tables, Seiten
	%\item TODO Custom Hyphenation
	%\item TODO Overfull/underfull Boxes
\end{itemize}

\begin{table}[ht]
\centering
\caption{Einbindung von Abkürzungen mit \texttt{glossaries}}
\label{tab:glossariesusage}
\begin{tabular}{ll} \toprule
\textbackslash{}gls\{\} & Normale Einbindung (lange Form beim ersten Auftreten)\\
\textbackslash{}Gls\{\} & Einbindung mit großem Anfangsbuchstaben\\
\textbackslash{}glspl\{\} & Pluralform (mit s oder wie angegeben)\\
\textbackslash{}glslink\{\}\{\} & Anzeige eines beliebigen anderen Textes\\
\bottomrule
\end{tabular}
\end{table}

\begin{figure}[ht]
\centering
\begin{tikzpicture}

% points
\path (90:4.5cm)    node[shape=circle,draw] (Scissors) {Scissors};
\path (90+1*72:4.5cm) node[shape=circle,draw] (Paper) {Paper};
\path (90+2*72:4.5cm) node[shape=circle,draw] (Rock) {Rock};
\path (90+3*72:4.5cm) node[shape=circle,draw] (Lizard) {Lizard};
\path (90+4*72:4.5cm) node[shape=circle,draw] (Spock) {Spock};

% edges
\tikzstyle{every node}=[above,sloped];

\draw[->,thick] (Scissors) -- node {cut} (Paper);
\draw[->,thick] (Paper) -- node {covers} (Rock);
\draw[->,thick] (Rock) -- node {crushes} (Lizard);
\draw[->,thick] (Lizard) -- node {poisons} (Spock);
\draw[->,thick] (Spock) -- node {smashes} (Scissors);
\draw[->,thick] (Scissors) -- node {decapitate} (Lizard);
\draw[->,thick] (Lizard) -- node {eats} (Paper);
\draw[->,thick] (Paper) -- node {disproves} (Spock);
\draw[->,thick] (Spock) -- node {vaporizes} (Rock);
\draw[->,thick] (Rock) -- node {crushes} (Scissors);

\end{tikzpicture}
\caption{Rock, Paper, Scissors, Lizard, Spock}
\label{fig:rpsls}
\end{figure}



\subsection{Vorgehensweise}
Folgende Punkte sind weiterhin zu beachten:
\begin{itemize}
	\item Es finden regelmäßige Konsultationen mit dem Betreuer oder den Betreuern statt. Grundsätzlich wird ein Serientermin mit zweiwöchentlichem Abstand vereinbart.
	\item Während des Schreibens der Diplom- oder Belegarbeit sollen dem Betreuer regelmäßig im \gls{SVN} des Lehrstuhls Zwischenstände der Ausarbeitung zur Verfügung gestellt werden. Der aktuelle Stand ist dem Betreuer ein bis zwei Tage vor der Konsultation zur Verfügung zu stellen, damit entsprechendes Feedback gegeben werden kann.
	\item Die Verteidigung der Diplom- oder Belegarbeit soll als Beamer-Präsentation durchgeführt werden.
	\item Vor dem Druck sollte die Option \texttt{\textbackslash{}printoutput} aktiviert werden (ganz vorn im Quelltext), damit die bunten Links schwarz werden.
	\item Die Arbeit soll einseitig gedruckt und gebunden sowie als PDF-Dokument abgegeben werden. In der Regel sind zwei gebundene und unterschriebene Exemplare einzureichen. Bei Abschlussarbeiten (Diplomarbeit, Masterarbeit, Bachelorarbeit) muss die Arbeit im Prüfungsamt vor der Einreichung abgestempelt werden.\todo{Eine seitliche Notiz mit \texttt{\textbackslash{}todo\{\}}}
	\item Es müssen die richtigen Fragen \cite{smartquestions} gestellt werden.
	\item Weitere nützliche Hinweise sind unter \cite{seusdiplomfaq} zu finden.
\end{itemize}
\todo[inline]{Eine Inline-Notiz mit \texttt{\textbackslash{}todo[inline]\{\}}}

\subsection{Editoren}
\LaTeX{}-Quelldokumente verwenden als Textdokumente einen Textzeichensatz. Hier kommt mit UTF-8 eine sehr weit verbreitete Umsetzung des global anwendbaren Standards \emph{Unicode} zum Einsatz. Dadurch können Sprachinkompatibilitäten vermieden werden. Diverse \LaTeX{}-Editoren kommen derzeit nicht mit bestimmten bzw. verschiedenen Zeichenkodierungen zurecht.\footnote{Übersicht von verbreiteten Editoren: \url{http://de.wikipedia.org/wiki/LaTeX}} Als guter Editor für viele Plattformen kann \emph{TeXMaker} empfohlen werden.



%\end{appendix}

\end{document}